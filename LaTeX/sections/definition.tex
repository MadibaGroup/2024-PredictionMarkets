% !TEX root = ../main.tex

We start with an abstract definition of a prediction market. Our definition generalizes on recent computer science-based definitions~\cite{BMR17,FPW23,SGKS25} which each presume a specific sub-types of a prediction market (\eg \cite{BMR17,SGKS25} use \textit{merging/splitting} while \cite{FPW23} presumes \textit{automated bookmaking} from \S\ref{wf:trade}). If our definitions are not clear, we refer the reader to Appendix~\ref{app:example} where we describe a specific market offered by Polymarket and map each part of definitions to this real world example.

\begin{definition}[Market]\label{def:market}
A (single) market is a tuple $M=(E,\Omega,J,R)$, where $E$ is a well-defined uncertain event, $\Omega$ is a nonempty outcome space for $E$, $J$ is a finite index set of contract labels (“shares”), and $R=(R_j)_{j\in J}$ are nonnegative payoff functions with $R_j:\Omega\to\mathbb{R}_{\ge 0}$. When $M$ resolves to $\omega_M\in\Omega$, one unit of share $j\in J$ pays $R_j(\omega_M)$ units of $\mathcal{N}$ (see Def. \ref{def:system}).
\end{definition}

\paragraph{Remark: WTA special-case.}
For a market $M=(E,\Omega,J,R)$, suppose there exists a bijection $\iota:J\to\Omega$ and
$R_j(\omega)\in\{0,1\}$ with $\sum_{j\in J} R_j(\omega)=1$ for all $\omega\in\Omega$.
Then $M$ is a winner-take-all (WTA) or Arrow--Debreu market: a unit claim of label $j$ pays $1$ iff the realized outcome equals $\iota(j)$, and $0$ otherwise.

\begin{definition}[Prediction Market System]\label{def:system}
A prediction--market system is a tuple $\mathcal{S}=(\mathcal{M},\mathcal{N},\mathsf{Res})$, where $\mathcal{M}$ is a countable set of markets, $\mathcal{N}$ is a numeraire (unit of account), and $\mathsf{Res}=\{\mathrm{res}_M\}_{M\in\mathcal{M}}$ is a family of resolution registers such that, for each $M$ with outcome space $\Omega_M$, we have $\mathrm{res}_M\in\{\bot\}\cup\Omega_M$, $\mathrm{res}_M$ is initially $\bot$, and $\mathrm{res}_M$ transitions exactly once to some $\omega_M\in\Omega_M$.
\end{definition}

% = = = 

\begin{definition}[System Axioms]\label{def:system} For every market $M=(E,\Omega,J,R)\in\mathcal{M}$ operating in system $\mathcal{S}=(\mathcal{M},\mathcal{N},\mathsf{Res})$, the following axioms hold.

\begin{enumerate}

\item \textbf{Issuance and Solvency.}
Let $S_j(M)\ge 0$ be the outstanding supply of label $j\in J$, and let $\mathsf{Treas}_M\ge 0$ be the market’s treasury (in the numeraire $\mathcal{N}$). The system maintains, at all times,
\[
  \mathsf{Treas}_M \;\ge\; \sup_{\omega\in\Omega_M}\; \sum_{j\in J} S_j(M)\,R_j(\omega).
\]
Thus when $q\ge 0$ new shares of any $j\in J$ are issued, treasury is increased sufficiently:
\[
  \Delta \mathsf{Treas}_M \;\ge\; \sup_{\omega\in\Omega_M}\; q\,R_j(\omega).
\]

\item \textbf{Transfer.} Holdings of each label $j\in J$ are transferable between accounts and transfers conserve per–label totals $S_j(M)$.
 
\item \textbf{Fungibility.} For each $j\in J$, shares are indistinguishable: for any $\omega\in\Omega_M$ and $q\ge 0$,
redeeming $q$ units yields $q\,R_j(\omega)$.
  
  
  \item \textbf{Settlement and redemption.} The resolution register satisfies $\mathrm{res}_M\in\{\bot\}\cup\Omega$, is initially $\bot$, and transitions exactly once\footnote{Real world \depms like Polymarket might resolve a market, receive a dispute of over the outcome, and resolve it differently after a process (see Section~\ref{wf:close}). In the definition, resolution refers to the final outcome only. An outcome is final when shares can be redeemed for payouts.} to a realized outcome $\omega_M\in\Omega$. Afterward, any holder of $q\ge 0$ units of label $j\in J$ may redeem them for $q\,R_j(\omega_M)$ units of $\mathcal{N}$. Redemption decreases outstanding shares $S_j(M)'\leftarrow S_j(M)-q$ and debits the treasury $\mathsf{Treas}_M'\leftarrow \mathsf{Treas}_M - q\,R_j(\omega_M)$, preserving the solvency invariant above.
 
 \end{enumerate}
 \end{definition}

% = = =


